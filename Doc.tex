\documentclass{article}

\usepackage[parfill]{parskip}  %  Include a blank line between paragraphs.

\sloppy
\pretolerance=5000
\tolerance=5000
\hyphenpenalty=10000
\exhyphenpenalty=10000
%                                          APA     My max   My min
%                                         default    page     page
\setlength{\footnotesep}    {   5mm}    %  5mm        5mm      5mm
\setlength{\topmargin}      { -15mm}    %  0mm      -25mm    -15mm
\setlength{\oddsidemargin}  {  -6mm}    %  0.25in    -6mm     -6mm
\setlength{\evensidemargin} {  -6mm}    %  0.25in    -6mm     -6mm
\setlength{\textwidth}      { 165mm}    %  6in      165mm    165mm
\setlength{\textheight}     { 240mm}    %  8.5in    240mm    240mm
\setlength{\headheight}     {   3mm}    %  ??         3mm      3mm
\setlength{\headsep}        {  10mm}    %  ??        10mm     10mm
\setlength{\footskip}       {  10mm}    %  ??        10mm     10mm


\title{fminsearcharb: Minimize Function With Constrained and/or Integer Parameters}
\author{Jeff Miller (miller at psy.otago.ac.nz)}

\begin{document}

\maketitle
\tableofcontents

\section{Overview}

fminsearcharb is a wrapper function that allows MATLAB users to employ fminsearch with constrained parameters.
It was inspired by John D'Errico's fminsearchbnd and fminsearchcon, but it allows a bit
more flexibility with regard to parameter constraints (e.g., parameters can be
constrained to be integers).

\section{Requirements}

As far as I know, it should work with any version of MATLAB that includes fminsearch.

\section{License}

Copyright (C) 2017, Jeff Miller

This program is free software: you can redistribute it and/or modify
it under the terms of the GNU General Public License as published by
the Free Software Foundation, either version 3 of the License, or
(at your option) any later version.

This program is distributed in the hope that it will be useful,
but WITHOUT ANY WARRANTY; without even the implied warranty of
MERCHANTABILITY or FITNESS FOR A PARTICULAR PURPOSE.  See the
GNU General Public License for more details.

You should have received a copy of the GNU General Public License
along with this program.  If not, see http://www.gnu.org/licenses/ 

\section{Syntax}

fminsearcharb is called with one of these basic parameter configurations:
\begin{itemize}
\item xparms2 = fminsearcharb(ErrFun,x0parms,realstoparms,parmstoreals)
\item xparms2 = fminsearcharb(ErrFun,x0parms,realstoparms,parmstoreals,parmcodes)
\item xparms2 = fminsearcharb(ErrFun,x0parms,realstoparms,parmstoreals,parmcodes,fmsoptions)
\item xparms2 = fminsearcharb(ErrFun,x0parms,realstoparms,parmstoreals,parmcodes,fmsoptions,d1,d2,\ldots)
\item {[xparms2,fval,exitflag,output]} = fminsearcharb(ErrFun,x0parms,...)
\end{itemize}

These are the input parameters (those marked * are exactly the same as in MATLAB's fminsearch):
\begin{itemize}
\item ErrFun*:  The error function to be minimized
\item x0parms*: A vector of the starting values for each parameter
\item realstoparms: A function mapping a vector of any real numbers onto a set of parameter values
                    satisfying your desired constraints.  For further details, see section~\ref{sec:constraints}.
\item parmstoreals: A function which is the inverse of realstoparms (i.e., maps your parameter values
                    onto the full set of real numbers considered by fminsearch).
\item parmcodes: A list of characters---one per parameter---indicating whether each parameter is
                 an adjustable real number, an adjustable integer, or fixed.
                 For further details, see section~\ref{sec:parmcodes}.
\item fmsoptions*: A set of options passed through to fminsearch.
\item d1,d2,\ldots: Additional \emph{data parameters} that are to be passed through to the ErrFunc
\end{itemize}

The output results are the same as those produced by MATLAB's fminsearch and are described in more
detail in their documentation.  In brief:
\begin{itemize}
\item xparms2: best parameter values that were found
\item fval: minimum value found for ErrFun
\item exitflag: the reason for stopping, returned as an integer.  1 = converged, 0 = Too many iterations,  -1 = terminated by output function.
\item output: structure holding information about the optimization process (e.g., number of iterations).
\end{itemize}

\section{Constraints}
\label{sec:constraints}

Example: Suppose you want to minimize an error function ErrFunc that depends on parameters p1, p2, and p3.
If p1, p2, and p3 can take on any real number values, you can just use fminsearch.
But suppose that there are three constraints---for example:
\begin{itemize}
\item $0<p1<1$
\item $100<=p2$
\item $p3>=p1+p2$
\end{itemize}

To use fminsearcharb, you must supply two functions that map back and forth between the
parameter ranges that you want and the full range of real numbers that fminsearch considers.

One function, called ``realstoparms'', maps from the range of all reals to the
range of the parameter space you want.
For the above three example constraints, this function could be:
\begin{verbatim}
function [ xparms ] = realstoparms( xreals, ~ )
% Example function to convert arbitrary reals to acceptable parm values
% according to the constraints in the documentation.
xparms = zeros(3,1);
xparms(1) = abs(xreals(1)) / (1 + abs(xreals(1)));   % xparms(1) always between 0 and 1
xparms(2) = 100 + xreals(2)^2;                       % xparms(2) always >= 100
xparms(3) = xparms(1) + xparms(2) + xreals(3)^2;     % xparms(3) always >= xparms(1) + xparms(2)
end
\end{verbatim}
Note that the output values for this mapping function always satisfy the desired
constraints for any set of real numbers that fminsearch might consider.

Of course you will have different constraints and so you must write a different
realstoparms function to implement those constraints.
There are almost always several different ways to write this function (i.e.,
different ways to ``enforce the constraints''), and it is up to you to choose
the one that works best for your problem.
(In my experience, it doesn't usually matter much.)

You must also write an inverse function, called ``parmstoreals'', that goes back
from the range of your parameter space to the range of all reals.
For the realstoparms function given above, the reverse function is:
\begin{verbatim}
function [ xreals ] = parmstoreals( xparms, ~ )
% Example function to convert acceptable parm values to arbitrary reals,
% exactly reversing the mapping used by realstoparms
xreals = zeros(3,1);
xreals(1) = xparms(1) / (1 - xparms(1));              % Assume 0<xparms(1)<1
xreals(2) = sqrt(xparms(2) - 100);                    % Assume xparms(2) >= 100
xreals(3) = sqrt(xparms(3) - xparms(1) - xparms(2));  % Assume xparms(3) >= xparms(1) - xparms(2)
end
\end{verbatim}

Included with fminsearcharb is a class called NumTrans, which contains some handy routines
that can be used within realstoparms and parmstoreals (see demo2.m for an example).

Here are several important points to remember about realstoparms and parmstoreals:
\begin{itemize}
\item Both functions must accept two input parameters.  The first is the vector of parms
      or reals that is to be converted.
      The second is the parmcodes string (see section~\ref{sec:parmcodes}).
      This string is passed to these functions by fminsearcharb, so the functions must
      be able to accept the second parameter even if it is not used.
      (It is rarely a good idea to have realstoparms and parmstoreals
      check parmcodes, but once in a while it is essential.)
\item These functions should process \emph{all} parameters, including any fixed ones.
\item These functions should process all parameters as real numbers and \emph{not}
      attempt to force any parameters to be integers (e.g., by rounding).
      For example, if you want a parameter to be an integer greater than or equal to 1,
      realstoparms and parmstoreals should just make sure it is greater than 1,
      but not round it.
      Conversion to integers happens as part of the search process within fminsearcharb.
      Doing that conversion within realstoparms and parmstoreals will disrupt the search.
\end{itemize}


\section{ParmCodes: Integer Parameters and Fixed Parameters}
\label{sec:parmcodes}

fminsearcharb also allows you to constrain parameters in two other ways
besides those provided by realstoparms and parmstoreals:
(a) You can constrain the parameter value to be an integer, or
(b) you can constrain it to remained fixed.
These constraints are specified by means of an fminsearcharb calling
parameter called ``parmcodes''.

Parmcodes is a string with one character for each of the parameters
that is passed to your error function.
Each character should be one of these possibilities:
\begin{itemize}
\item r: This parameter should be adjusted by fminsearch, and it can be any real number, subject to the constraints
         enforced by realstoparms.
\item i: This parameter should be adjusted by fminsearch, and it must be an integer, subject to the constraints
         enforced by realstoparms.
\item f: This parameter should not be adjusted by fminsearch.
\end{itemize}

\section{Other Issues}

All of the considerations involving function minimization (e.g., local minima, multiple solutions) apply as usual,
and these are sometimes worsened if there are integer parameters.

The use of transformations can increase these problems and introduce additional ones.
For an excellent discussion of these issues, see the document ``Understanding fminsearchbnd'' by John D'Errico
in his FMINSEARCHBND contribution to MATLAB File Exchange.
The sections  ``Multiple solutions due to the transformations''
and  ``Starting values, infeasible starting values, tolerances, etc.''
are particularly relevant to fminsearcharb.


\section{Search Method With Integer Parameters}

For anyone who is curious, this section briefly describes fminsearcharb's method of searching for integer parameters.
You don't need to read this section if you just want to use fminsearcharb.

fminsearch always tries real-valued parameters as candidates to minimize ErrFun.
So, for an integer parameter, fminsearcharb evaluates the error function twice, once for
each integer value above and below the current real-valued candidate (which I admit may
not be very efficient).
For example, if fminsearch wants to evaluate the candidate value of 6.25, fminsearcharb
evaluates the error function with floor(6.25)=6 and ceil(6.25)=7.
The overall error function computed for 6.25 (and returned to fminsearch) is computed as
the weighted average of the error functions at 6 and 7, with weights of 0.75 and 0.25,
respectively (6 gets the higher weight because 6.25 is closer to 6).

Note that this procedure always converges onto some integer as long as two adjacent integers give
different error values.
For example, it will go to 7 if ErrFun(6) $>$ ErrFun(7), because the smallest weighted average error value
is 0.0*ErrFun(6)+1.00*ErrFun(7).

Note also that this procedure naturally tries larger and smaller integers as appropriate.
For example, if ErrFun(6) $>$ ErrFun(7), fminsearch might suggest a series of values like
6.25, 6.65, 6.90, 7.10, \ldots.
Once the candidate value of 7.10 is reached, the new overall error will be the weighted average
of ErrFun(7) and ErrFun(8), so the function will ``find'' 8 if that is better than 7.
And so on.

When searching with 2,3,\ldots\ integer parameters, the same basic idea is used in 2,3,\ldots\ dimensions.

\end{document}
